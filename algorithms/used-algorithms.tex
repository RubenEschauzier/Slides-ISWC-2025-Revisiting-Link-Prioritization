
% NOTES:
%   - Use figures to explain the RCC and Intermediate prioritization algorithms. Its not clear now; too much text
\begin{frame}{Prioritization algorithms \parencite{hartig2016walking}}
  \begin{block}{Non-adaptive}
    \begin{itemize}
        \item Breadth-first (default), depth-first, random prioritization
    \end{itemize}
  \end{block}
  \begin{block}{Graph-based}
        \begin{itemize}
        \item In-degree, PageRank score
    \end{itemize}
  \end{block}
  \begin{block}{Result-based}
        \begin{itemize}
        \item Uses result contribution count (RCC) of each node (URI)
        \item Priority equal to the sum / count of non-zero RCC of the 1 or 2-hop in-neighbours of a node
        \item Called \emph{rcc-1}, \emph{rcc-2}, \emph{rel-1}, \emph{rel-2} respectively. 
    \end{itemize}
  \end{block}
\end{frame}

\begin{frame}{Prioritization algorithms}
  \begin{block}{Intermediate-results}
        \begin{itemize}
        \item Uses intermediate solutions in the engine
        \item Priority of URI equal to the largest intermediate result with that URI bound
        \item \emph{IS} sets initial priorities to 0, while \emph{ISdcr} sets priority to the priority of the parent node - 1
    \end{itemize}
  \end{block}
  \begin{block}{Hybrid}
        \begin{itemize}
        \item Multiply intermediate and full result scoring functions
        \item \emph{is-rcc1}, \emph{is-rcc2}, \emph{is-rel1}, \emph{is-rel2} 
    \end{itemize}
  \end{block}
  \begin{block}{TypeIndex}
        \begin{itemize}
        \item TypeIndex points to location for resource of specific type
        \item Prioritize TypeIndex
    \end{itemize}
  \end{block}
\end{frame}

\begin{frame}[t]{Prioritization algorithms}
  \begin{block}{Oracle}
        \begin{itemize}
        \item Compute RCC in hindsight
        \item Scores are propegated through the shortest path
        \item Serves as optimal performance oracle
    \end{itemize}
  \end{block}
\end{frame}
